\subsection{Explicación del algoritmo implementado}
La heurística golosa que implementamos consiste en buscar el nodo de mayor grado del grafo.
A partir de ese nodo v, se construye una clique con sus nodos adyacentes de la siguiente manera:

\begin{algorithm}[H]
\caption{Goloso}\label{ej2}
\begin{algorithmic}[1]
\Procedure{Goloso}{$G=(V,E)$}
	\State clique  $\shortleftarrow$ $\{$v nodo de mayor grado$\}$
	\State frontera $\shortleftarrow$ $\{$vecinos de v$\}$
	\While{ $\{Aumente frontera\}$ }
		\State $\{$Buscar nodo u$\}$ tal que u $\in$ frontera y $ d(u) \ge d(p)$ $ \forall p \neq u $ y forme clique con nodos de clique
		\State clique $\cup$ $\{u\}$
		\State frontera $\cup$ $\{$vecinos de u$\}$ sacando repetidos y nodos de la clique
	\EndWhile
	\State return 
\EndProcedure
\end{algorithmic}
\end{algorithm}
