\subsection{Explicación del algoritmo implementado}

Para resolver el problema de CMF utilizaremos el algoritmo de Bron-Kerbosch con algunas modificaciones. Dado un grafo G, el algoritmo de Bron-Kerbosch busca todas las cliques maximales de G utilizando una técnica de backtracking recursivo. Más generalmente, dados tres conjuntos R, P y X, R forma cliques recolectando nodos de P y X. P le brinda posibles nodos candidatos a R para armar cliques mientras que el conjunto X se conforma de vértices que ya fueron utilizados previamente para la extensión de la clique R y que por lo tanto son ignorados para formar nuevas cliques. Para cada nodo $v$ en P se hace una llamada recursiva en donde $v$ es agregado a R y, P y X se restringen a todos los nodos adyacentes a $v$. Luego, se mueve el nodo $v$ de P al conjunto X y se continúa con un nuevo vértice de P para formar nuevas cliques.

La recursión se inicia seteando a R y a X como conjuntos vacíos y a P como el conjunto conformado por todos los nodos del grafo. Por cada llamado recursivo, el algoritmo chequea si los conjuntos P y X se encuentran vacíos y, en caso de estarlos, reporta a R como clique maximal. Sin embargo, nuestro algoritmo no realiza el último paso descripto ya que el objetivo es encontrar la clique de máxima frontera. En vez de verificar si los conjuntos P y X están vacíos, calcula el cardinal de la frontera de la clique formada en R y, en caso de ser mayor que la encontrada hasta ese momento de ejecución, la guarda. A continuación un pseudocódigo del algoritmo:

\begin{algorithm}[H]
\caption{findCliques}\label{ej1}
\begin{algorithmic}[1]
\Procedure{findCliques}{$R,P,X$}
	\State int $front$ $\shortleftarrow$ frontera(R)
	\If {front $>$ maxFrontera}
		\State cliqueMaxFrontera $\shortleftarrow$ R
		\State maxFrontera $\shortleftarrow$ front
	\EndIf
	\ForAll{$p\ vertice \in\ P$}
		\State findCliques(R $\cup$ $\{v\}$, P $\cap$ N($v$), X $\cap$ N($v$))
	\EndFor
	\State P $\backslash$ $\{v\}$
	\State X $\cup$ $v$
\EndProcedure
\end{algorithmic}
\end{algorithm}

Algunas aclaraciones:
\begin{itemize}
	\item N(v) es el conjunto de todos los nodos adyacentes a $v$ en el grafo.
	\item frontera(R) calcula el cardinal de la frontera de la clique R.
	\item $cliqueMaxFrontera$ guarda los nodos de la clique de máxima frontera y $maxFrontera$ almacena el cardinal de la frontera de dicha clique.
\end{itemize}

\subsubsection{Ejemplo de ejecución}

\begin{figure}[H]
 \centering
  \subfloat[]{
   \label{}
    \includegraphics[scale=0.3]{exacto/imgs/ejemplito.png}}
  \subfloat[Frontera=4.]{
   \label{}
    \includegraphics[scale=0.3]{exacto/imgs/ejemplitoSol.png}}
\caption{Un ejemplo y su solución}

\end{figure}

\begin{center}
    \begin{tabular}{ | l | l | l | l | l | p{5cm} |}
    \hline
    Llamada recursiva & R & P & X & cliqueMaxFrontera & Comentarios \\ \hline
    1 & $\{$ $\}$ & $\{$1,2,3,4,5,6$\}$ & $\{$ $\}$ & $\{$ $\}$ & Candidatos: todos los nodos del grafo. \\ \hline
    2 & $\{$1$\}$ & $\{$2$\}$ & $\{$ $\}$ & $\{$1$\}$ Frontera: 1 & Formo la clique R y el único candidato posible es el nodo 2 (único adyacente).\\ \hline
    3 & $\{$1,2$\}$ & $\{$ $\}$ & $\{$ $\}$ & $\{$1,2$\}$ Frontera: 2 & La clique R no puede agregar más nodos. Los adyacentes a 2 no son adyacentes a 1. \\ \hline
    4 & $\{$2$\}$ & $\{$3,4$\}$ & $\{$1$\}$ & $\{$2$\}$ Frontera: 3 & La clique R tiene dos candidatos para agregar (sus adyacentes) pero no puede agregar el nodo 1 que se encuentra en X ya que esa clique ya fue analizada previamente.\\ \hline
    5 & $\{$2,3$\}$ & $\{$4$\}$ & $\{$ $\}$ & $\{$2,3$\}$ Frontera: 4 & Candidato: 4. Adyacente a 2 y 3.\\ \hline
    6 & $\{$2,3,4$\}$ & $\{$ $\}$ & $\{$ $\}$ & $\{$2,3$\}$ Frontera: 4  &\\ \hline
    7 & $\{$2,4$\}$ & $\{$ $\}$ & $\{$ $\}$ & $\{$2,3$\}$ Frontera: 4 & El 3 no es candidato ya que fue evaluada esa opción previamente. \\ \hline
    8 & $\{$3$\}$ & $\{$4,5$\}$ & $\{$ $\}$ & $\{$2,3$\}$ Frontera: 4 & El 2 no es candidato ya que fue evaluada esa opción previamente.\\ \hline
    9 & $\{$3,4$\}$ & $\{$ $\}$ & $\{$ $\}$ & $\{$2,3$\}$ Frontera: 4 & \\ \hline
    10 & $\{$3,5$\}$ & $\{$ $\}$ & $\{$ $\}$ & $\{$2,3$\}$ Frontera: 4 & \\ \hline
    11 & $\{$4$\}$ & $\{$6$\}$ & $\{$ $\}$ & $\{$2,3$\}$ Frontera: 4 & \\ \hline
    12 & $\{$4,6$\}$ & $\{$ $\}$ & $\{$ $\}$ & $\{$2,3$\}$ Frontera: 4 & \\ \hline
    13 & $\{$5$\}$ & $\{$ $\}$ & $\{$ $\}$ & $\{$2,3$\}$ Frontera: 4 & \\ \hline
    14 & $\{$6 $\}$ & $\{$ $\}$ & $\{$ $\}$ & $\{$2,3$\}$ Frontera: 4 & \\
	\hline
    \end{tabular}
\captionof{table}{Seguimiento del algoritmo. Solución devuelta: $\{$2,3$\}$, frontera: 4.}
\end{center}

Notar que en el seguimiento mostrado el algoritmo evalúa todas las cliques posibles del grafo y sólo lo hace una vez por cada una de ellas. Esto ocurre gracias a la presencia de los conjuntos P y X.

\subsection{Complejidad}

Sea $G$ un grafo y $|C|$ la cantidad de cliques de $G$ podemos afirmar que el algoritmo exacto presentado tiene una complejidad lineal relativa a la cantidad de cliques presentes en $G$, es decir, $|C|$. Decimos que es linear ya que como fue mostrado en la sección 1.1.1, el algoritmo procesa todas las cliques presentes en $G$ y sólo lo hace una vez por cada una de ellas.

Podemos, entonces, asegurar que el número total de llamadas recursivas que hace el algoritmo es $|C|$. Suponiendo que en cada llamada recursiva tenemos un costo de $O(L)$, la complejidad total del algoritmo será $O(L.|C|)$. Resta analizar cuánto es $O(L)$ y $|C|$.

Para analizar $O(L)$ observemos el pseudocódigo del algoritmo:

n = cantidad de nodos en el grafo.

\begin{algorithm}[H]
\caption{findCliques}\label{ej1}
\begin{algorithmic}[1]
\Procedure{findCliques}{$R,P,X$}
	\State int $front$ $\shortleftarrow$ frontera(R) \comp{n}
	\If {front $>$ maxFrontera}
		\State cliqueMaxFrontera $\shortleftarrow$ R \comp{n}
		\State maxFrontera $\shortleftarrow$ front \comp{1}
	\EndIf
	\ForAll{$p\ vertice \in\ P$}
		\State findCliques(R $\cup$ $\{v\}$, P $\cap$ N($v$), X $\cap$ N($v$)) \Comment{complejidad analizada para la unión e intersección} \comp{n+n^2+n^2}
	\EndFor
	\State P $\backslash$ $\{v\}$ \comp{n}
	\State X $\cup$ $v$ \comp{n}
\EndProcedure
\end{algorithmic}
\end{algorithm}

\begin{algorithm}[H]
\caption{frontera}\label{ej1}
\begin{algorithmic}[1]
\Procedure{frontera}{$R$}
	\State int $res$ $\shortleftarrow$ 0 \comp{1}
	\State int $tam$ $\shortleftarrow$ $|R|$ \comp{1}
	\ForAll{$n\ vertice \in\ R$} \comp{n}
		\State $res$ $\shortleftarrow$ $res \ + \ adyacentes(n)$ \comp{1}
	\EndFor
	\State $res \shortleftarrow res \ - \ tam*(tam-1)$ \comp{1}
	\State return $res$ \comp{1}
\EndProcedure
\end{algorithmic}
\end{algorithm}

\begin{itemize}
 \item frontera(R) es $O(n)$.
 \item R $\cup$ $\{v\}$ es $O(n)$ ya que el conjunto tiene a lo sumo n elementos y para verificar que $v$ no se encuentre en el conjunto hay que recorrerlo todo.
 \item P $\cap$ N($v$) es $O(n^2)$ ya que los dos conjuntos a intersecar tienen a lo sumo n elementos y por cada elemento en el primer conjunto debe recorrerse el segundo conjunto para verificar si el elemento se encuentra en él.
\end{itemize}

Por lo tanto $O(L)$ = $O(n^2)$. Veamos cuánto es $|C|$.

% No estoy seguro de lo que voy a poner abajo. De hecho dudo de que sea cierto. Verificar

Llegamos a la conclusión de que un grafo de n vértices tiene a lo sumo $2^n$ cliques que es equivalente a la cardinalidad del conjunto de partes de un conjunto de n elementos. ¿Por qué decimos esto? Porque para nosotros un $K_n$ es un grafo cuya cantidad de cliques es máximo para un grafo de n elementos por el hecho de que cualquier subconjunto de $K_n$ es una clique posible ya que todos los nodos se encuentran interconectados.

En definitiva, la complejidad total del algoritmo es: $O(n^2*2^n)$.

\subsubsection{Performance del Algoritmo}



